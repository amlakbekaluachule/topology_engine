\documentclass[11pt]{article}

\usepackage{amsmath, amssymb, amsthm}
\usepackage{geometry}
\usepackage{hyperref}
\usepackage{listings}
\usepackage{setspace}

\geometry{margin=1in}
\setstretch{1.15}

\title{\textbf{Topology Engine}}
\author{Amlakbekalu Achule \\ Columbia University \\ Computer Science \& Statistics}
\date{}

\begin{document}
\maketitle

\section*{Overview}

Topology Engine is a computational playground for algebraic topology.
It models topological spaces, loops, and homotopy using exact algebraic structures rather than numerical approximation.

The project translates standard textbook definitions into executable objects.

\section*{Functionality}

TopoPlay allows the user to:
\begin{itemize}
    \item Define simple topological spaces (graphs, surfaces)
    \item Represent loops as algebraic objects
    \item Compute homotopy classes via word reduction
    \item Work with fundamental groups $\pi_1$
    \item Model basic covering spaces
    \item Experiment with surfaces of arbitrary genus
\end{itemize}

All reasoning is definition-driven and algebraic.

\section*{Mathematical Model}

\subsection*{Loops and Homotopy}

A loop based at a point is represented as a word in generators of the fundamental group.

Two loops $\gamma_1$ and $\gamma_2$ are homotopic if and only if their reduced words coincide:
\[
\gamma_1 \simeq \gamma_2 \quad \Longleftrightarrow \quad [\gamma_1] = [\gamma_2] \in \pi_1(X).
\]

Homotopy is implemented via algebraic word reduction.

\subsection*{Free Groups}

For a wedge of $n$ circles,
\[
\pi_1\left(\bigvee_{i=1}^{n} S^1\right) \cong F_n,
\]
where $F_n$ is the free group on $n$ generators.

TopoPlay implements:
\begin{itemize}
    \item generators,
    \item inverses,
    \item concatenation,
    \item reduction by cancellation.
\end{itemize}

The fundamental reduction rule is:
\[
g g^{-1} = e.
\]

\subsection*{Surfaces}

An orientable surface of genus $g$ is modeled using generators
\[
a_1, b_1, \dots, a_g, b_g
\]
subject to the defining relation
\[
\prod_{i=1}^{g} [a_i, b_i] = e,
\]
where the commutator is defined as
\[
[a, b] = a b a^{-1} b^{-1}.
\]

This relation is explicitly constructed and used to test null-homotopy.

\subsection*{Fundamental Group Examples}

\paragraph{Circle}
\[
\pi_1(S^1) \cong \mathbb{Z}.
\]

\paragraph{Wedge of two circles}
\[
\pi_1(S^1 \vee S^1) \cong F_2.
\]

\paragraph{Torus}
\[
\pi_1(T^2) = \langle a, b \mid a b a^{-1} b^{-1} = e \rangle.
\]

\section*{Design Principles}

\begin{itemize}
    \item Algebra over geometry
    \item Definitions over heuristics
    \item Exact reasoning over numerical approximation
    \item Code mirrors standard topology texts
\end{itemize}

\section*{Limitations}

\begin{itemize}
    \item No automatic subgroup enumeration
    \item Covering spaces are conceptual, not exhaustive
\end{itemize}

\section*{Motivation}

The goal of this project is to explore how abstract concepts from algebraic topology can be encoded directly into software without reducing them to numerical simulation.

\section*{Possible Extensions}

\begin{itemize}
    \item van Kampen's theorem
    \item Classification of surfaces
    \item Explicit covering space construction
    \item Simplicial or cubical complexes
\end{itemize}

\end{document}
